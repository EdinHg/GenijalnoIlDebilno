\sloppy
\chapter{NEFUNKCIONALNI ZAHTJEVI}

\sloppy
\section*{Kontrola verzija}

Zerina Ahmetović - atributi kvalitete

\sloppy
\section{Atributi kvalitete}
%po 4 za svaku subs tip-dostupnost npr, opis i vrsta-kvalit.ili kvant, task2
\sloppy
\subsection{Operativni atributi}

\begin{enumerate}
    \item\textbf{Tip:} dostupnost\\
    \textbf{Opis:} Sistem mora biti dostupan 24/7 jer korisnici mogu kupovati karte i pregledati predstave u bilo koje doba dana.\\
    \textbf{Vrsta:} kvantitativni
    \item\textbf{Tip:} dostupnost\\
    \textbf{Opis:} Ažuriranja sistema trebaju se vršiti noću u periodu 00:00–05:00, uz minimalno vrijeme nedostupnosti, koristeći postupne \textit{deploy} strategije (npr. \textit{blue-green deployment}).\\
    \textbf{Vrsta:} kvantitativni
    \item\textbf{Tip:} pristupačnost\\
    \textbf{Opis:} Sistem mora omogućiti brz pristup informacijama o predstavama, gdje je učitavanje manje od 2 sekunde po stranici).\\
    \textbf{Vrsta:} kvantitativni
    \item\textbf{Tip:} sigurnost\\
    \textbf{Opis:} Podaci o korisnicima i transakcijama moraju biti zaštićeni enkripcijom (npr. TLS za komunikaciju, AES za podatke).\\
    \textbf{Vrsta:} kvalitatitvni
    \item\textbf{Tip:} upotrebljivost\\
    \textbf{Opis:} Sistem treba imati jednostavno i jasno korisničko uputstvo koje se prikazuje pri prvom korištenju ili na zahtjev.\\
    \textbf{Vrsta:} kvalitatitvni
    \item\textbf{Tip:} povjerljivost\\
    \textbf{Opis:} Sistem prikuplja osnovne korisničke podatke (ime, \textit{email}, historija kupovine), koji moraju biti čuvani u skladu s pravilima o privatnosti.\\
    \textbf{Vrsta:} kvalitatitvni
    \item\textbf{Tip:} povjerljivost\\
    \textbf{Opis:} Korisnik mora dati saglasnost prilikom registracije da se njegovi podaci mogu čuvati i koristiti za potrebe sistema.\\
    \textbf{Vrsta:} kvalitatitvni
\end{enumerate}

\sloppy
\subsection{Revizijski atributi}

\begin{enumerate}
    \item\textbf{Tip:} održivost\\
    \textbf{Opis:} Osoba koja održava sistem mora poznavati osnovne \textit{DevOps} principe (npr. \textit{CI/CD, Docker, Git}), kako bi bila sposobna za nadogradnje bez prekida rada.\\
    \textbf{Vrsta:} kvalitativni
    \item\textbf{Tip:} fleksibilnost\\
    \textbf{Opis:} Sistem koristi troslojnu arhitekturu (prezentacija, poslovna logika, pristup podacima) i modularni dizajn, što omogućava zamjenu ili nadogradnju pojedinih dijelova bez uticaja na ostatak sistema.\\
    \textbf{Vrsta:} kvalitativni
    \item\textbf{Tip:} fleksibilnost\\
    \textbf{Opis:} Fleksibilnost se dodatno osigurava korištenjem REST API-ja i nezavisnih frontend-backend komponenti.\\
    \textbf{Vrsta:} kvalitativni
    \item\textbf{Tip:} skalabilnost\\
    \textbf{Opis:} Sistem mora podržati skaliranje za do 5x povećanja broja korisnika u periodima visokog opterećenja (kulturne sezone, festivali, promocije), pri čemu mora zadržati vrijeme odziva ispod 2 sekunde za 95\% zahtjeva.\\
    \textbf{Vrsta:} kvantitativni
    \item\textbf{Tip:} skalabilnost\\
    \textbf{Opis:} \textit{Backend} koristi \textit{Node.js} koji omogućava \textit{event-driven }model, idealan za skalabilne aplikacije.\\
    \textbf{Vrsta:} kvalitativni
    \item\textbf{Tip:} jednostavnost\\
    \textbf{Opis:} Sistem je dizajniran da bude kompleksan po mogućnostima, ali jednostavan za korištenje i održavanje – svaka komponenta ima jasnu funkciju i dokumentovani interfejs.\\
    \textbf{Vrsta:} kvalitativni
\end{enumerate}

\sloppy
\subsection{Tranzicijski atributi}

\begin{enumerate}
    \item\textbf{Tip:} portabilnost\\
    \textbf{Opis:} Sistem koristi \textit{web} tehnologije koje su platformski nezavisne, te aplikacija radi na svim modernim pregledačima i operativnim sistemima \textit{(Windows, macOS, Linux, Android, iOS}), kao i na mobilnim uređajima i tabletima.\\
    \textbf{Vrsta:} kvalitativni
    \item\textbf{Tip:} interoperabilnost\\
    \textbf{Opis:} Uska grla mogu nastati kod plaćanja (latencija i fallback greške) te kod integracija sa društvenim mrežama ako promijene API-je, stoga je implementirana modularnost i mogućnost zamjene konekcija.\\
    \textbf{Vrsta:} kvalitativni
    \item\textbf{Tip:} interoperabilnost\\
    \textbf{Opis:} Sistem mora ostvariti komunikaciju sa platformama za plaćanje (\textit{PayPal}, \textit{Monri}) putem API-ja i sa društvenim mrežama putem njihovih SDK-ova i REST API-ja za dijeljenje sadržaja i autentikaciju.\\
    \textbf{Vrsta:} kvalitativni
    \item\textbf{Tip:} ponovna upotrebljivost\\
    \textbf{Opis:} Prepravke korisničkog interfejsa ili zamjena domena (npr. iz „pozorište“ u „koncerti“) zahtijevaju minimalne izmjene jer je logika sistema odvojena od sadržaja.\\
    \textbf{Vrsta:} kvalitativni
\end{enumerate}

\sloppy
\section{Ograničenja}

\sloppy
\subsection{Finansijska ograničenja}

Opisati sva finansijska ograničenja i na koji način utječu na projekat.

\sloppy
\subsection{Hardverska ograničenja}

Opisati sva hardverska ograničenja i na koji način utječu na projekat.

\sloppy
\subsection{Tehnološka ograničenja}

Opisati sva tehnološka ograničenja i na koji način utječu na projekat.

\sloppy
\subsection{Organizacijska ograničenja}

Opisati sva organizacijska ograničenja i na koji način utječu na projekat.

\sloppy
\subsection{Operativna ograničenja}

Opisati sva operativna ograničenja i na koji način utječu na projekat.

\sloppy
\subsection{Zakonska ograničenja}

Opisati sva zakonska ograničenja i na koji način utječu na projekat.